% -*-LaTex-*-

%-----------------------------------------------------------------------------
%   Copyright 2016 Florian Schumacher (Ruhr-Universitaet Bochum, Germany)
%
%   This file is part of the ASKI manual as a LaTeX document with main file
%   manual.tex
%
%   Permission is granted to copy, distribute and/or modify this document
%   under the terms of the GNU Free Documentation License, Version 1.3
%   or any later version published by the Free Software Foundation;
%   with no Invariant Sections, no Front-Cover Texts, and no Back-Cover Texts.
%   A copy of the license is included in the section entitled ``GNU
%   Free Documentation License''. 
%-----------------------------------------------------------------------------
%
This chapter is intended to guide you, dependent on what workflow you want to follow, through all necessary steps
to achieve your goals. Make sure that you have read section entitled ``How to get started''. 

If you don't know about \ASKI{} yet, we recommend you to quickly read through the previous section 
entitled ``What is \ASKI{}?'', which explains some basic terminology in \ASKI{} and the concepts it is based on.

The sections below address possible operations you can conduct with \ASKI{}. For every operation, we only refer 
to the necessary basic steps (by $\rightarrow$), which are described in chapter~\myref{basic_steps}.\\
Make sure to have a complete read-through before hastily doing anything!

Good Luck!
%
%
%++++++++++++++++++++++++++++++++++++++++++++++++++++++++++
\newpage
\section{Full Waveform Inversion using pre-integrated spectral Waveform Sensitivity Kernels} \label{guide,sec:classic_inversion}
%\phantomsection  % so hyperref creates bookmarks
%\addcontentsline{toc}{section}{Full Waveform Inversion - Classical Waveform Sensitivity Kernels}
%++++++++++++++++++++++++++++++++++++++++++++++++++++++++++
%
%
%here short introduction as to what ``Full Waveform Sensitivity Kernel Inversion'' actually is, followed by
%the steps to do:

Iterative inversion scheme which uses waveform sensitivity kernels to gain model updates form data residua.

You should have already installed \ASKI{}, see section entitled ``How to get started''. 
In addition to \ASKI{}, you will need to install a supported software which 
solves the forward problem \myaref{basic_steps,sec:forward_problem}.
%
%----------------------------------------------------------
\subsection*{Before The First Iteration Step}
%----------------------------------------------------------
%
\begin{itemize}
\item Create a main parameter file (e.g.\ in the parent directory of your specific inversion directory, 
or where you collect main parameter files for all your inversion projects)
\myaref{basic_steps,sec:main_parfile}. You will need this file as an input argument to almost all programs/scripts.\\
Set \lcode{MAIN_PATH_INVERSION} to a correct value. 
The directory does not need to exist yet, if not, then it will be created. \\
You should keep the default values of \lcode{ITERATION_STEP_PATH} and \lcode{PARFILE_ITERATION_STEP} unless
you know what you are doing. If you wish to use different values for the complete inversion process, 
it makes senese to adjust them now.\\
All other parameters can be adjusted later. 
%
\item Create a directory structure for the expected number of iteration steps of your inversion 
\myaref{basic_steps,sec:create_dir}
%
\item Make yourself familiar with the form of data used in \ASKI{} \myaref{basic_steps,sec:data_general}. \\
  Set \lcode{APPLY_EVENT_FILTER}, \lcode{APPLY_STATION_FILTER}, \lcode{PATH_MEASURED_DATA}, 
  \lcode{PATH_EVENT_FILTER}, \lcode{PATH_STATION_FILTER}, 
  \lcode{FILE_EVENT_LIST}, and \lcode{FILE_STATION_LIST}, as well as \lcode{MEASURED_DATA_FREQUENCY_STEP}, 
  \lcode{MEASURED_DATA_NUMBER_OF_FREQ} and \lcode{MEASURED_DATA_INDEX_OF_FREQ}  in your main 
  parameter file, before preparing your data in the required form \myaref{basic_steps,sec:measured_data}, 
  as well as any filters \myaref{basic_steps,sec:filters}.
%
\item Set \lcode{FORWARD_METHOD} in your main 
parameter file to the value of your choice. If you want to use different methods in the course of 
inverting one dataset (e.g.\ starting with a 1D method, continuing with a 3D method), then it may make 
sense to create a different directory structure for each method and using the final model of one method
as the starting model for the next method.
%
\item Choose a model parametrization by setting \lcode{MODEL_PARAMETRIZATION} in the main parameter file 
to a value of your choice (which is supported by your forward method).
%
\end{itemize}
%
%----------------------------------------------------------
\subsection*{Before Each Iteration Step (including the first one)}
%----------------------------------------------------------
%
\begin{itemize}
\item Set \lcode{CURRENT_ITERATION_STEP}
in your main parameter file to the correct value. When continuing your inversion with a different method, you
may also keep the current iteration step index (in order for you not to get confused) and leave 
subdirectories of your \lcode{MAIN_PATH_INVERSION} empty (or delete them after creation if they 
are not needed):\\
e.g.\ an inversion with one method could start with iteration step 4 (and respective subdirectory), 
if you have already conducted 3 iteration steps with other methods.
%
\item Define the inversion grid of the current iteration \myaref{basic_steps,sec:invgrid}.
%
\item Set all parameters in the specific iteration step parameter file to correct values 
\myaref{basic_steps,sec:iter_parfile}, including the correct reference to your inversion grid.
Refer to the documentation of your forward method on how to set filenames \lcode{FILE_WAVEFIELD_POINTS}
and \lcode{FILE_KERNEL_REFERENCE_MODEL} (as the handling of these file are method dependent).
%
\item Dependent on your method and model parametrization, take care about communicating the current
model (inverted in the previous iteration) to your forward method \myaref{basic_steps,sec:export_kim}. 
Before the first iteration, however, you need to define some starting model \myaref{basic_steps,sec:start_model}.
%
\end{itemize}
%
%----------------------------------------------------------
\subsection*{Conducting An Iteration Step}
%----------------------------------------------------------
%
\begin{itemize}
\item Compute forward wavefields and Green tensor components w.r.t.\ the current model 
by your method. Refer to the respective documentation of your method.\\
After that, you may prepare the synthetic data in the way \ASKI{} expects it (see sections~\myref{basic_steps,sec:data_general} 
and~\myref{files,sec:synth_data}). Refer to the documentation of your method on how to do it.
%
\item Set \lcode{TYPE_INTEGRATION_WEIGHTS} in the iteration specific parameter file 
\myaref{basic_steps,sec:intw}.
You should keep the default values of \lcode{FILE_INTEGRATION_WEIGHTS},
unless you know what you are doing (can be any name, will be created, but is referred to, afterwards. 
You could compute different types of weights and store them in different files, this way). 
%
\item Initiate basic requirements for all programs and scripts \myaref{basic_steps,sec:initBasics}.
%
\item Define data and model space, where the paths are mainly important for now \myaref{basic_steps,sec:dmspace}.
%
\item Compute sensitivity kernels for your specific set of paths and your set of model parameters 
  \myaref{basic_steps,sec:compute_kernels}. \\
  If desired, you may have a look at your kernels \myaref{basic_steps,sec:plot_kernels}.
%
\item Choose a specific data and model space. You may well play around with different subsets of data or
smoothing (next step) in the course of inverting for different models \myaref{basic_steps,sec:dmspace}.
%
\item Finally compute the inverted model by solving the kernel system \myaref{basic_steps,sec:solve_kernel_system},
  possibly varying the regularization constraints.
%
\item After deciding whether or not to do another iteration of full waveform inversion and choosing a model 
  for the next iteration \myaref{basic_steps,sec:investigate_convergence}, repeat all operations beginning 
  with ``Before Each Iteration Step (including the first one)''.
\end{itemize}
%
%
%++++++++++++++++++++++++++++++++++++++++++++++++++++++++++
\newpage
\section{Toy Example Using \lcodetitle{SPECFEM3D\_Cartesian} as a forward solver} \label{guide,sec:example_SPEC_C}
%\phantomsection  % so hyperref creates bookmarks
%\addcontentsline{toc}{section}{Toy Example Using \lcodetitle{SPECFEM3D\_Cartesian} as a forward solver}
%++++++++++++++++++++++++++++++++++++++++++++++++++++++++++
%
%
Please refer to section~\myref{guide,sec:example_C_borehole}.
%
%
%++++++++++++++++++++++++++++++++++++++++++++++++++++++++++
\newpage
\section{Toy Example Using \lcodetitle{SPECFEM3D\_GLOBE} as a forward solver} \label{guide,sec:example_SPEC_GLOB}
%\phantomsection  % so hyperref creates bookmarks
%\addcontentsline{toc}{section}{Toy Example Using \lcodetitle{SPECFEM3D\_GLOBE} as a forward solver}
%++++++++++++++++++++++++++++++++++++++++++++++++++++++++++
%
%
This section describes a very short example of computing a spherical waveform kernel for just one event-station 
pair using the setting of example "regional\_Greece\_small" from the \lcode{SPECFEM3D_GLOBE 7.0.0} release package 
(example uses 4 CPU cores only).

The kernel files of this example package were computed with \ASKI{} version 1.0, which however should
be reproducable also with higher versions of \ASKI{}
since the changes to version 1.2 do not effect the 
computations or compatibility of the output files. It may, however, 
occur that numbers might not be reproduced in a numerically exact way, which is also likely to happen
when a different compiler (version) is used. 

You should have already installed the \ASKI{} main package, see section entitled ``How to get started''. 

Then download the example
package \lcode{ASKI_example_spherical_small.tar.gz} which is attached to release \lcode{v1.0} 
of gitHub repository \url{https://github.com/seismology-RUB/ASKI} (198 MB, probably go to 
\url{https://github.com/seismology-RUB/ASKI/releases/tag/v1.0}).
The wavefield files in this example package were produced by version \lcode{1.0} of the extension package
\lcode{SPECFEM3D_GLOBE for ASKI} 
(\url{https://github.com/seismology-RUB/SPECFEM3D_GLOBE_for_ASKI/releases/tag/v1.0}), 
but you should be able to re-produce them as well with version \lcode{1.2} of the extension package
(\url{https://github.com/seismology-RUB/SPECFEM3D_GLOBE_for_ASKI/releases/tag/v1.2}).

\emph{Please note} that this example is not documented as detailed as the 
Cartesian cross-borehole example~\myaref{guide,sec:example_C_borehole}. Even if you do not intent to use 
the forward code \lcode{SPECFEM3D_Cartesian}, you are still advised to have a look at the Cartesian
cross-borehole example (or get to know \ASKI{} otherwise), in order to practice the operations of the main \ASKI{}
package and, thus, learn about how to conduct any steps \emph{after} the forward problem is solved, in particular
how to compute kernels. Otherwise, refer to the general FWI workflow until the point of kernel 
computation~\myaref{guide,sec:classic_inversion} in order to re-produce the kernels provided in this 
example package.

\emph{Most importantly}, you must change the value of \lcode{MAIN_PATH_INVERSION}
in file\\
\lcode{ASKI_example_spherical_small/main_parfile} to
\lcode{"your_base_path/"} in order for all the example directory to work. 
If you do not want to overwrite the existing output, alternatively you 
should create a separate inversion directory from scratch~\myaref{basic_steps,sec:create_dir}
and leave \lcode{ASKI_example_spherical_small/}
as a reference only. 
%
%
%++++++++++++++++++++++++++++++++++++++++++++++++++++++++++
\newpage
\section{Cross Borehole Example} \label{guide,sec:example_C_borehole}
%\phantomsection  % so hyperref creates bookmarks
%\addcontentsline{toc}{section}{Cross Borehole Example}
%++++++++++++++++++++++++++++++++++++++++++++++++++++++++++
%
%
This section describes the synthetic cross borehole example presented in Florian's dissertation 
\cite{_743d334d-dfa4-4a16-8cc5-91cdadc95271} (chapter 5.1) as well as in our GJI paper \cite{Schumacher16} 
(section 4.1). It is an example of lull waveform inversion using pre-integrated spectral waveform sensitivity 
kernels (as described in the first recipe~\myref{guide,sec:classic_inversion} of this chapter. 

The example files referred to in the following, were computed with \ASKI{} version 1.0, which however should
not pose any problem since the changes from version 1.0 to 1.2 do not effect the computations or compatibility
of any output files. It may, however, 
occur that numbers might not be reproduced in a numerically exact way, which is also likely to happen
when a different compiler (version) is used. 

You should have already installed \ASKI{}, see section entitled ``How to get started''. 

Then download the example
package \lcode{ASKI_inversion_cross_borehole.tar.gz} which is attached to release \lcode{v1.0} 
of gitHub repository \url{https://github.com/seismology-RUB/ASKI} (566 MB, probably go to 
\url{https://github.com/seismology-RUB/ASKI/releases/tag/v1.0}).
In order to get started with this example package, you may use the commands in script
\lcode{ASKI_inversion_cross_borehole_run.sh}, provided for download at the very same location
(\emph{please adjust variable} \lcode{ASKI} \emph{at the beginning of the script!}).

In this package, all files of the first 2 iterations (plus illustrating extras) are provided. 
In order to keep the download small, wavefield output is not provided  for
iterations 3 to 12. You should be able to reproduce the wavefields immendiately for any iteration step
from the provided inverted models without reproducing all preceeding iterations. 
The pre-integrated sensitivity kernel files are only provided for
iterations 1 to 3. Kernel files for the other iterations can be downloaded by
interested users seperately as package 
\lcode{ASKI_inversion_cross_borehole_sensitivity_kernels_iter04-iter12.tar.gz}
(907 MB, same location \url{https://github.com/seismology-RUB/ASKI/releases/tag/v1.0}).

\emph{Most importantly}, you must change the value of \lcode{MAIN_PATH_INVERSION}
in file\\
\lcode{ASKI_inversion_cross_borehole/main_parfile_cross_borehole} to
\lcode{"your_base_path/"} in order for all the example directory to work. 
If you do not want to overwrite the existing output, alternatively you 
should create a separate inversion directory from scratch~\myaref{basic_steps,sec:create_dir}
 and leave \lcode{ASKI_inversion_cross_borehole/}
as a reference only (this strategy is followed by script \lcode{ASKI_inversion_cross_borehole_run.sh}. 

This example was generated using \lcode{SPECFEM3D_Cartesian 3.0} for \ASKI{} version 1.0.
(\url{https://github.com/seismology-RUB/SPECFEM3D_Cartesian_for_ASKI/releases/tag/v1.0}).
Even if you intend
to use a different forward method, you can still practice the operations of the main \ASKI{}
package and, thus, learn about how to conduct any steps in
the full waveform inversion \emph{after} the forward problem is solved. Just read the description
of the contained files in \\
\lcode{ASKI_inversion_cross_borehole/README}.

If you want to reproduce also the forward wavefields from \lcode{SPECFEM3D_Cartesian 3.0}, 
you need to install it, of course~\myaref{basic_steps,sec:forward_problem}.
%
%
%++++++++++++++++++++++++++++++++++++++++++++++++++++++++++
\newpage
\section{Time-Domain Sensitivity Kernels} \label{guide,sec:time_kernels}
%\phantomsection  % so hyperref creates bookmarks
%\addcontentsline{toc}{section}{Time-Domain Sensitivity Kernels}
%++++++++++++++++++++++++++++++++++++++++++++++++++++++++++
%
%
This section describes, how to compute time-domain waveform sensitivity kernels for a specific set 
of source-receiver paths with respect to a certain background earth model as an operation seperate of any
other \ASKI{} operations. The kernels in time-domain are much more intuitive to look at for human beings, 
than the standard frequency-domain sensitivity kernels. You may, as well, compute time-domain sensitivity 
kernels from the kernels produced in any iteration step of a full waveform inversion 
(page \pageref{guide,sec:classic_inversion}). For this purpose, apply the steps 
``Transforming to Time-Domain Sensitivity Kernels'' (below) after you computed the spectral kernels in 
your iteration step, as the time-domain waveform kernels are produced by an inverse Fourier transform 
from the standard frequency-domain waveform sensitivity kernels on which \ASKI{} is based.

Please do not get confused by the general terminology of \emph{inversion} and \emph{iteration}, etc. 
Technically you will be conducting an incomplete first iteration step of a full waveform inversion, 
using all the program infrastructure which is also used for a full waveform inversion. 

You should have already installed \ASKI{}, see section entitled ``How to get started''. 
In addition to \ASKI{}, you will need to install a supported software which 
solves the forward problem \myaref{basic_steps,sec:forward_problem}.
%
%----------------------------------------------------------
\subsection*{Preceding Considerations}
%----------------------------------------------------------
%
\begin{itemize}
\item Create a main parameter file (e.g.\ in the parent directory of your specific inversion directory, 
or where you collect main parameter files for all your inversion projects or analyses)
\myaref{basic_steps,sec:main_parfile}. You will need this file as an input argument to almost all programs/scripts.\\
Set \lcode{MAIN_PATH_INVERSION} to a correct value.
The directory does not need to exist yet, if not, then it will be created. \\
Set \lcode{ITERATION_STEP_PATH} and 
\lcode{PARFILE_ITERATION_STEP} to desired values, 
or leave the default values, if present.\\
All other parameters can be adjusted later. 
%
\item Create a directory structure for only one iteration step \myaref{basic_steps,sec:create_dir}
%
\item Even if you do not have any measured data, it might still be beneficial for you to make yourself 
(roughly) familiar with the form of data used in \ASKI{} \myaref{basic_steps,sec:data_general}. \\
In your main parameter file, set the following values: 
  \begin{itemize}
  \item Set \lcode{FILE_EVENT_LIST} and \lcode{FILE_STATION_LIST} to define the sources and receivers which are
    involved in the paths that you would like to compute the kernels for. 
  \item Dependent on the (length of the) time series you want to deal with, define the frequency discretization of the
    spectral kernels that will be produced first. This must be done by \lcode{MEASURED_DATA_FREQUENCY_STEP}, 
    \lcode{MEASURED_DATA_NUMBER_OF_FREQ} and \lcode{MEASURED_DATA_INDEX_OF_FREQ}.
  \end{itemize}
In general, for the pure kernel computation you do not need any measured data. So, here you do not need to prepare
data in the \ASKI{} required form.
%
\item Set \lcode{APPLY_EVENT_FILTER}, \lcode{APPLY_STATION_FILTER} (and if required \lcode{PATH_EVENT_FILTER} and 
  \lcode{PATH_STATION_FILTER}) in your main parameter file. In case the forward wavefields were calculated w.r.t.\ an
  impulse source time function (or the source-time function was deconvolved), you should apply filters 
  \myaref{basic_steps,sec:filters} that taper down the amplitude spectrum before the maximum frequency used. 
  Otherwise, the inverse Fourier transform below can create artefacts.
%
\item Set \lcode{FORWARD_METHOD} in your main parameter file to the value of your choice. 
%
\item Choose a model parametrization by setting \lcode{MODEL_PARAMETRIZATION} in the main parameter file 
to a value of your choice (which is supported by your forward method).
%
\item Set \lcode{CURRENT_ITERATION_STEP} in your main parameter file to value \lcode{1}, as you are technically
starting to conduct the first (and only) iteration step of a full waveform inversion.
%
\item It is highly recommended to set \lcode{DEFAULT_VTK_FILE_FORMAT} to \lcode{BINARY} in the main parameter file, 
  since a lot of \lcode{vtk} files might be generated (one for every time step). Using the binary format significantly
  reduces storage.

\item Define the inversion grid \myaref{basic_steps,sec:invgrid}, which controls the spacial volumetric discretization 
(resolution) of the computed sensitivity kernels. In case of just computing (time) kernels to look at, it is not
crucial to regard this resolution as the resolution of some inverted model, as no inversion will be conducted on the 
inversion grid. Since there might be a lot of \lcode{vtk} files generated (one for every time step), it is highly recommended to set\\
\lcode{VTK_GEOMETRY_TYPE = CELL_CENTERS} in the inversion grid parameter file (if your type of inversion grid
supports that functionality). This significantly reduces the amount of geometry information written to the 
\lcode{vtk} files, hence their size.
%
\item Set all parameters in the specific iteration step parameter file to correct values 
\myaref{basic_steps,sec:iter_parfile}, including the correct reference to the inversion grid. Set 
\lcode{ITERATION_STEP_NUMBER_OF_FREQ} and \lcode{ITERATION_STEP_INDEX_OF_FREQ} to the \emph{same} values as the
\lcode{MEASURED_DATA_NUMBER_OF_FREQ} and \lcode{MEASURED_DATA_INDEX_OF_FREQ} in the main parameter file!
Refer to the documentation of your forward method on how to set filenames \lcode{FILE_WAVEFIELD_POINTS}
and \lcode{FILE_KERNEL_REFERENCE_MODEL} (as the handling of these file are method dependent).
%
\item Dependent on your method and model parametrization, define your background model with respect to which the
kernels will be computed. If you have some inverted model file, use \myaref{basic_steps,sec:export_kim}. For defining
a starting model, see \myaref{basic_steps,sec:start_model}.
%
\end{itemize}
%
%----------------------------------------------------------
\subsection*{Computing Spectral Waveform Sensitivity Kernels}
%----------------------------------------------------------
%
\begin{itemize}
\item Compute forward wavefields and Green tensor components w.r.t.\ the current model 
by your method. Refer to the respective documentation of your method.\\
After that, you may prepare the synthetic data in the way \ASKI{} expects it (see sections~\myref{basic_steps,sec:data_general} 
and~\myref{files,sec:synth_data}). Refer to the documentation of your method on how to do it.
%
\item Set \lcode{TYPE_INTEGRATION_WEIGHTS} in the iteration specific parameter file 
\myaref{basic_steps,sec:intw}. If you intend to compute the time kernels on wavefield points only, you will technically
never use the integration weights, but nevertheless they need to be correctly created (in that case use a simple
type of weights like 5 or 0).
You should keep the default values of \lcode{FILE_INTEGRATION_WEIGHTS},
unless you know what you are doing (can be any name, will be created, but is referred to, afterwards. 
You could compute different types of weights and store them in different files, this way). 
%
\item Initiate basic requirements for all programs and scripts \myaref{basic_steps,sec:initBasics}.
%
\item If you have many paths, you may define a data and model space concentrating on defining paths 
  \myaref{basic_steps,sec:dmspace}.
  If you have only one path or just a few, it is possible (and propably also convenient) to just continue to the
  computation of the kernels.
%
\item Compute the standard frequency-domain sensitivity kernels for your specific set of paths (or the one or 
  few paths, one after another). In case you want to transform to time-domain kernels \emph{on wavefield points} 
  (and not pre-integrated kernels on inversion grid),
  use option \lcode{-wp} \myaref{basic_steps,sec:compute_kernels} \\
  If desired, you may have a look at the standard frequency-domain kernels \myaref{basic_steps,sec:plot_kernels}.
\end{itemize}
%
%----------------------------------------------------------
\subsection*{Transforming to Time-Domain Sensitivity Kernels}
%----------------------------------------------------------
%
\begin{itemize}
\item Transform the standard frequency-domain waveform kernels to time domain. In case you want to get  
  time-domain kernels \emph{on wavefield points}, (and not pre-integrated kernels on inversion grid),
  you must use option \lcode{-wp} (in this case, you already should have used \lcode{-wp} to compute 
  spectral kernels) \myaref{basic_steps,sec:compute_time_kernels}.\\
  Make sure that the wavefield spectra have a sufficiently small amplitude spectrum at high frequencies, in order for the
  inverse Fourier Transform to work satisfactorily. Otherwise use filters.
%
\item Plot the time kernels \myaref{basic_steps,sec:plot_time_kernels}.
\end{itemize}
%
%++++++++++++++++++++++++++++++++++++++++++++++++++++++++++
\newpage
\section{Analysis of Acquisition Geometry using Kernel Focussing} \label{guide,sec:acq_ana_focus}
%\phantomsection
%\addcontentsline{toc}{section}{Analysis of Acquisition Geometry using Kernel Focussing}
%++++++++++++++++++++++++++++++++++++++++++++++++++++++++++
%
{\bf THIS SECTION IS WORK IN PROGRESS!}

%Short introduction as to what ``Analysis of Acquisition Geometry using Kernel Focussing'' actually is:
The general concept of Backus-Gilbert-based focussing of sensitivity kernels on a defined focussing region 
in the model space is published in \cite{_743d334d-dfa4-4a16-8cc5-91cdadc95271}, Chapter 7.1. On the basis
of this way of finding linear combinations of data which are sensitive only to changes in a model subspace, 
the dataset can be analysed w.r.t.\ finding event-station paths (and components / frequencies) which optimally
illuminate the model subspace. 

Executable \lcode{focusSpectralKernels} \myaref{programs_scripts,sec:bin_prog,sec:focus_spec_kernel} produces the coefficients of the linear combination of data required
for the analysis. However, this section does not yet give a detailed list of steps how to conduct such 
an analysis. 

\begin{itemize}
\item step 1
\item step 2
\item ...
\end{itemize}
%
%++++++++++++++++++++++++++++++++++++++++++++++++++++++++++
\newpage
\section{Full Waveform Inversion - Focused Waveform Sensitivity Kernels} \label{guide,sec:focused_inversion}
%\phantomsection
%\addcontentsline{toc}{section}{Full Waveform Inversion - Focused Waveform Sensitivity Kernels}
%++++++++++++++++++++++++++++++++++++++++++++++++++++++++++
%
{\bf THIS SECTION IS WORK IN PROGRESS!}

%Short introduction as to what ``Full Waveform Inversion using Focused Data for Maximum Resolution'' actually is:
The general concept of Backus-Gilbert-based focussing of sensitivity kernels on a defined focussing region 
in the model space is published in \cite{_743d334d-dfa4-4a16-8cc5-91cdadc95271}, Chapter 7.1. On the basis
of this way of finding linear combinations of data which are sensitive only to changes in a model subspace, 
different disjoint subdivisions of the model space can be chosen, finding the finest subdivision that
is resolved by the data (i.e.\ where focussing of sensitivity on each subdivision cell is still successfull).
This gives the finest model space that the data can resolve. Also this gives a set of ``new'' data consisting of
linear combinations of the original data, for which each of this new data is sensitive only to a certain
spot in the model space. Doing inversion steps of full waveform inversion with the focussed kernels and the 
focussed data, the hope is to overall conduct a waveform inversion at the maximum resolving power of the data. 
This can also be seen as a sophisticated form of regularizing the overall inverse problem. 

Such a ``focussed waveform inversion'' still remains an idea and was not yet approached to be implemented
in \ASKI{}. Please don't hesitate to implement and test this concept.

todo list:
\begin{itemize}
\item step 1
\item step 2
\item ...
\end{itemize}

