% -*-LaTex-*-

%-----------------------------------------------------------------------------
%   Copyright 2016 Florian Schumacher (Ruhr-Universitaet Bochum, Germany)
%
%   This file is part of the ASKI manual as a LaTeX document with main file
%   manual.tex
%
%   Permission is granted to copy, distribute and/or modify this document
%   under the terms of the GNU Free Documentation License, Version 1.3
%   or any later version published by the Free Software Foundation;
%   with no Invariant Sections, no Front-Cover Texts, and no Back-Cover Texts.
%   A copy of the license is included in the section entitled ``GNU
%   Free Documentation License''. 
%-----------------------------------------------------------------------------
%
%% ATTENTION NOTE FOR USERS ACCIDENTLY GETTING STARTED WITH ASKI (in a hurry) WITHOUT
%% FOLLOWING THE INTENDED PROCEDURE HOW TO READ THIS MANUAL
\inotice{This chapter is \emph{not} intended to be read through item by item (as a manual)! If you are new to \ASKI{} and
  accidently looked up this chapter in the hope to find what you're looking for, you are strongly advised to
  quickly read section ``How to get started'' at the beginning of this document, explaining how to use this manual.}

This chapter collects documentation on file formats involved in \ASKI{}.
%
%++++++++++++++++++++++++++++++++++++++++++++++++++++++++++
\section{Parameter Files} \label{files,sec:parfiles}
%++++++++++++++++++++++++++++++++++++++++++++++++++++++++++
%
Parameter files are simple text files.

The following type of lines are ignored:
\begin{itemize}
\item comment lines, i.e.\ lines STARTING with an arbitrary number of blanks followed by a ``\#'' character
\item empty lines and lines containing blanks only
\item lines not containing any ``='' character
\end{itemize}

How to specify one parameter:
\begin{itemize}
\item valid lines have the form ``keyword = value'' (blanks leading or following ``keyword'', ``='', or ``value'' are ignored)
\item in a valid line, all characters in front of ``='' (without leading and appending blanks)
are interpreted as the keyword, allowing for blank characters within the keyword (e.g.\ for lines  
\mbox{``\hspace{5mm}key~word~=~value\hspace{3mm}''}, the string ``key~word'' is used as the keyword)
\item all characters behind ``='' (without leading and appending blanks) are interpreted as the value string from 
which the value is read, which in particular means that ``\#'' comments at the end of a line (such as 
\mbox{``\hspace{3mm}keyword = value\hspace{2mm}\# comment\hspace{3mm}''}) are \emph{not} allowed!
\end{itemize}
By convention, specify \emph{paths} (i.e.\ directory names, which will be concatenated with a filename of 
a file in that directory) always ending on ``/'' and specify \emph{filenames} always \emph{without} leading ``/''.
%
%----------------------------------------------------------
\subsection{Main Parameter File} \label{files,sec:main_parfile}
%----------------------------------------------------------
%
Here, shortly all keywords required in the main parameter file for your specific program operation, are described.
%- - - - - - - - - - - - - - - - - - - - - - - - - - - - - 
\subsubsection{\lcode{FORWARD_METHOD}} \label{files,sec:main_parfile,itm:forward_method}
\begin{itemize}
\item[] \lcode{GEMINI}
\item[] \lcode{SPECFEM3D} $\rightarrow$ \lcode{SPECFEM3D_Cartesian} and \lcode{SPECFEM3D_GLOBE}
\item[] \lcode{NEXD} 
\end{itemize}
For details on the methods and references to their documentation, refer to section~\ref{basic_steps,sec:forward_problem}
%- - - - - - - - - - - - - - - - - - - - - - - - - - - - -
\subsubsection{\lcode{MODEL_PARAMETRIZATION}} \label{files,sec:main_parfile,itm:mod_pmtrz}
\begin{itemize}
\item[] \lcode{isoLameSI} $\rightarrow$ isotropic Lam\'e parameters and density in SI units: 
    $\rho$ (\lcode{rho}) [kg/m\textsuperscript{3}], $\lambda$ (\lcode{lambda}) [Pa], $\mu$ (\lcode{mu})  [Pa]
\item[] \lcode{isoVelocitySI} $\rightarrow$ isotropic seismic velocities and density in SI units:
    $\rho$ (\lcode{rho}) [kg/m\textsuperscript{3}], $v_p$ (\lcode{vp}) [m/s], $v_s$ (\lcode{vs}) [m/s]
\item[] \lcode{isoVelocity1000} $\rightarrow$ isotropic seismic velocities and density in  
  units involving a factor of 1000: $\rho$ (\lcode{rho}) [g/cm\textsuperscript{3}], $v_p$ (\lcode{vp}) [km/s], 
  $v_s$ (\lcode{vs}) [km/s]
\end{itemize}
No other parameterization supported yet.
%- - - - - - - - - - - - - - - - - - - - - - - - - - - - - 
\subsubsection{\lcode{PARAMETER_CORRELATION_MODE}} \label{files,sec:main_parfile,itm:par_cor_mode}
Any model parameters can be correlated to others which are not inverted for. So far supported correlation modes:
\begin{itemize}
\item[] \lcode{NONE} $\rightarrow$ no parameter correlation will be done whatsoever
\item[] \lcode{CORRELATE_KERNELS} $\rightarrow$ will add correlated kernel values of other parameters to the kernels
  of those parameters which are inverted for (experimental feature, not yet generally applicable in practice, 
  please refer to \ASKI{} developers manual for things ``to do'' in order to properly support this feature).
\end{itemize}
%- - - - - - - - - - - - - - - - - - - - - - - - - - - - -
\subsubsection{\lcode{PARAMETER_CORRELATION_FILE}} \label{files,sec:main_parfile,itm:par_cor_file}
File (relative to \lcode{MAIN_PATH_INVERSION}) containing the definition of the correlation and the correlation coefficients. 
The given filename is ignored in case of \lcode{PARAMETER_CORRELATION_MODE = NONE}.
%- - - - - - - - - - - - - - - - - - - - - - - - - - - - -
\subsubsection{\lcode{MAIN_PATH_INVERSION}} \label{files,sec:main_parfile,itm:main_path}
All subpaths for filenames are considered relative to this main path. This
directory is thought to contain all your relevant output and (temporary) data.\\
Example: \lcode{MAIN_PATH_INVERSION = /data/inversions/Aegean1/}
%- - - - - - - - - - - - - - - - - - - - - - - - - - - - -
\subsubsection{\lcode{CURRENT_ITERATION_STEP}} \label{files,sec:main_parfile,itm:cur_iter_step}
Example: \lcode{CURRENT_ITERATION_STEP = 3}
%- - - - - - - - - - - - - - - - - - - - - - - - - - - - - 
\subsubsection{\lcode{ITERATION_STEP_PATH}} \label{files,sec:main_parfile,itm:iter_path}
Relative to main path, defining name of subdirectory of \lcode{MAIN_PATH_INVERSION} which contains 
all relevant (meta)data of an inversion step. A three-digit integer (= \lcode{CURRENT_ITERATION_STEP}) 
and ``/'' will be appended to \lcode{ITERATION_STEP_PATH} (i.e.\ ``001/'', ``002/'', \dots) defining the 
first, second \dots iteration step directory.\\
Example (default): \lcode{ITERATION_STEP_PATH = iteration_step_}
%- - - - - - - - - - - - - - - - - - - - - - - - - - - - - 
\subsubsection{\lcode{PARFILE_ITERATION_STEP}} \label{files,sec:main_parfile,itm:iter_parfile}
File name of iteration step specific parameter file, relative to \lcode{MAIN_PATH_INVERSION/ITERATION_STEP_PATH}
Example: \lcode{PARFILE_ITERATION_STEP = iter_parfile}
%- - - - - - - - - - - - - - - - - - - - - - - - - - - - - 
\subsubsection{\lcode{PATH_MEASURED_DATA,APPLY_EVENT_FILTER,PATH_EVENT_FILTER,APPLY_STATION_FILTER,PATH_STATION_FILTER}} 
\label{files,sec:main_parfile,itm:path_mdata_filters}
Paths and flags related to the measured data. The paths are \emph{absolute}, i.e.\ can be everywhere, 
e.g.\ close to where you 
have stored/processed your (time domain) data, or in directory \lcode{MAIN_PATH_INVERSION}, etc.\ \dots

If you do not wish to use any filtering per events, you can switch to\\
\lcode{APPLY_EVENT_FILTER = .false.},\\
if you do not wish to use any fitering per stations, you can switch to\\
\lcode{APPLY_STATION_FILTER = .false.}.\\
In these cases, the respective filter files are not expected to exist and the respective filter paths are
ignored.

The naming convention of the files in the respective directories is:\\
\lcode{FILE_MEASURED_DATA}: \lcode{data_EVENTID_STATIONNAME_COMP},\\
\lcode{FILE_EVENT_FILTER}: \lcode{filter_EVENTID},\\
\lcode{FILE_STATION_FILTER}: \lcode{filter_STATIONNAME_COMP}, \\
where station filters are dependet on component and \lcode{STATIONNAME} and \lcode{EVENTID} are 
defined in \lcode{FILE_STATION_LIST} and \lcode{FILE_EVENT_LIST} file, and \lcode{COMP} 
is a valid component (for valid components see \ref{basic_steps,sec:data_general})\\
Example: \\
\lcode{PATH_MEASURED_DATA = /mydata/your_name_of_inversion/ASKI_data/}\\
\lcode{PATH_EVENT_FILTER = /mydata/your_name_of_inversion/ASKI_event_filter/}\\
\lcode{PATH_STATION_FILTER = /mydata/your_name_of_inversion/ASKI_station_filter/}\\

%- - - - - - - - - - - - - - - - - - - - - - - - - - - - - 
\subsubsection{\lcode{FILE_EVENT_LIST}} \label{files,sec:main_parfile,itm:file_event_list}
Absolute filename where \ASKI{} finds a text file defining a set of events in the required format 
(\ref{files,sec:event_list})\\
Example: \lcode{FILE_EVENT_LIST = /mydata/your_name_of_inversion/ASKI_events}
%- - - - - - - - - - - - - - - - - - - - - - - - - - - - - 
\subsubsection{\lcode{FILE_STATION_LIST}} \label{files,sec:main_parfile,itm:file_station_list}
Absolute filename where \ASKI{} finds a text file defining a set of stations in the required format 
(\ref{files,sec:station_list})\\
Example: \lcode{FILE_STATION_LIST = /mydata/your_name_of_inversion/ASKI_stations}
%- - - - - - - - - - - - - - - - - - - - - - - - - - - - - 
\subsubsection{\lcode{MEASURED_DATA_FREQUENCY_STEP,MEASURED_DATA_NUMBER_OF_FREQ,MEASURED_DATA_INDEX_OF_FREQ}} 
\label{files,sec:main_parfile,itm:mdata_freq}
Discretized frequency window of measured data (same expected in event\_filter/station\_filter!) given by a frequency 
step \lcode{FREQUENCY_STEP} [Hz] and a vector of frequency indices \lcode{INDEX_OF_FREQ}
(of length \lcode{NUMBER_OF_FREQ}), where for specific frequency index $i$ the corresponding frequency $f_i$ [Hz] 
computes to $f_i = i \cdot$ \lcode{FREQUENCY_STEP}\\
Example:\\
\lcode{MEASURED_FREQUENCY_STEP = 10.}\\
\lcode{MEASURED_NUMBER_OF_FREQ = 5}\\
\lcode{MEASURED_INDEX_OF_FREQ = 2 3 5 7 10}\\
which corresponds to the 5 frequencies $20,30,50,70,100$ Hz
%- - - - - - - - - - - - - - - - - - - - - - - - - - - - - 
\subsubsection{\lcode{UNIT_FACTOR_MEASURED_DATA}} 
Indicate the unit factor of the measured data. The definition of the factor is:\\
Multiplication of the measured data (displacement) with this factor gives values in the SI unit of meters [m].\\
Example: If you use seismic time-domain data for \ASKI{} in the unit of nano meters [nm], then \lcode{UNIT_FACTOR_MEASURED_DATA}
must be set to \lcode{1.0e-9} , because the data values in nano meters [nm] must be divided by $1.0\cdot 10^9$ in order to get meters [m].
If you use seismic displacement data in the SI unit of meters, then \lcode{UNIT_FACTOR_MEASURED_DATA} must be set to \lcode{1.0}.
%- - - - - - - - - - - - - - - - - - - - - - - - - - - - - 
\subsubsection{\lcode{DEFAULT_VTK_FILE_FORMAT}} 
Either \lcode{BINARY} or \lcode{ASCII} defining the default type of \lcode{vtk} files 
which will be produced in the course of running the programs.
%
%----------------------------------------------------------
\subsection{Parameter File for Specific Iteration Step} \label{files,sec:iter_parfile}
%----------------------------------------------------------
%
Here, shortly all keywords required in a parameter file for a specific iteration step, i.e.\ \\
\lcode{MAIN_PATH_INVERSION/ITERATION_STEP_PATH/PARFILE_ITERATION_STEP} , are described.

These parameter files are created automatically by script \lcode{py/create_ASKI_dir.py}. 
Alternatively, you can look at the template file \lcode{template/iter_parfile_template}.
%- - - - - - - - - - - - - - - - - - - - - - - - - - - - - 
\subsubsection{\lcode{USE_PATH_SPECIFIC_MODELS}}
Set the flag to \lcode{.true.} if you intend to use the \ASKI{} functionality of path specific kernel 
reference models (usually only for the very first iteration and a 1D method, as described in 
section~\myref{basic_steps,sec:path_specific}). For a regular iteration, i.e.\ using one 
global kernel reference model (for all paths, i.e.\ kernels) set this flag to \lcode{.false.}.
%- - - - - - - - - - - - - - - - - - - - - - - - - - - - - 
\subsubsection{\lcode{ITERATION_STEP_NUMBER_OF_FREQ, ITERATION_STEP_INDEX_OF_FREQ}} \label{files,sec:iter_parfile,itm:iter_freq}
Frequency discretization of this iteration step, must be a subset of global frequency discretization 
for this inversion defined as defined by \ref{files,sec:main_parfile,itm:mdata_freq}. \\
\lcode{ITERATION_STEP_NUMBER_OF_FREQ} must be smaller or equal to \lcode{MEASURED_DATA_NUMBER_OF_FREQ} and vector \\
\lcode{ITERATION_STEP_INDEX_OF_FREQ} (of length \lcode{ITERATION_STEP_NUMBER_OF_FREQ})
must only contain indices contained in \lcode{MEASURED_DATA_INDEX_OF_FREQ}.\\
All indices here are assumed in accordance with the global frequency step \lcode{MEASURED_DATA_FREQUENCY_STEP}.
%- - - - - - - - - - - - - - - - - - - - - - - - - - - - - 
\subsubsection{\lcode{TYPE_INVERSION_GRID, PARFILE_INVERSION_GRID}} \label{files,sec:iter_parfile,itm:invgrid}
Type of inversion grid (as supported, cf.\ \ref{basic_steps,sec:invgrid}) and corresponding
filename of parameter file defining this inversion grid, relative to \\
\lcode{MAIN_PATH_INVERSION/ITERATION_STEP_PATH/}
%- - - - - - - - - - - - - - - - - - - - - - - - - - - - - 
\subsubsection{\lcode{TYPE_INTEGRATION_WEIGHTS}}
Type of integration weights (integer number), cf.\ \ref{basic_steps,sec:intw} for supported values.
%- - - - - - - - - - - - - - - - - - - - - - - - - - - - - 
\subsubsection{\lcode{FILE_INTEGRATION_WEIGHTS}} 
Filename of the integration weights file, which will be created and used, relative to 
\lcode{MAIN_PATH_INVERSION}. 
%- - - - - - - - - - - - - - - - - - - - - - - - - - - - -
\subsubsection{\lcode{FILE_WAVEFIELD_POINTS}} 
Filename of the wavefield points file, relative to \lcode{MAIN_PATH_INVERSION}, which
is in general created by the method you are using. Just refer here to this file. 
%- - - - - - - - - - - - - - - - - - - - - - - - - - - - -
\subsubsection{\lcode{FILE_KERNEL_REFERENCE_MODEL}} \label{files,sec:iter_parfile,itm:model}
Dependent on the method you are using, these filenames may be handled individually. Please refer to the respective 
documentation of the methods for recommendations how to use these parameters, or which naming to choose.
%- - - - - - - - - - - - - - - - - - - - - - - - - - - - - 
\subsubsection{\lcode{FILEBASE_BASIC_STATS}}
Base filename of vtk stats output files (related to inversion grid, wavefield points, integration weights,
events, stations), relative to \lcode{MAIN_PATH_INVERSION/ITERATION_STEP_PATH}. 
%- - - - - - - - - - - - - - - - - - - - - - - - - - - - - 
\subsubsection{\lcode{PATH_OUTPUT_FILES}}
Folder relative to which some sensitivity analysis and inversion programs may write their output (relatively small output
like models, coefficients etc., NO wavefields/kernels etc.!), relative to 
\lcode{MAIN_PATH_INVERSION/ITERATION_STEP_PATH}.  Make sure the path ends on ``/".
%- - - - - - - - - - - - - - - - - - - - - - - - - - - - -
\subsubsection{\lcode{PATH_KERNEL_DISPLACEMENTS}} 
Subdirectory of current iteration step path
\lcode{MAIN_PATH_INVERSION/ITERATION_STEP_PATH} which contains the 
kernel displacement files. Make sure the path ends on ``/".
%- - - - - - - - - - - - - - - - - - - - - - - - - - - - -
\subsubsection{\lcode{PATH_KERNEL_GREEN_TENSORS}} 
Subdirectory of current iteration step path
\lcode{MAIN_PATH_INVERSION/ITERATION_STEP_PATH} which contains the 
kernel green tensor files. Make sure the path ends on ``/".
%- - - - - - - - - - - - - - - - - - - - - - - - - - - - -
\subsubsection{\lcode{PATH_SENSITIVITY_KERNELS}} 
Subdirectory of current iteration step path
\lcode{MAIN_PATH_INVERSION/ITERATION_STEP_PATH} which contains the 
velocity kernel files. Make sure the path ends on ``/".
%- - - - - - - - - - - - - - - - - - - - - - - - - - - - -
\subsubsection{\lcode{PATH_SYNTHETIC_DATA}} 
Subdirectory of current iteration step path
\lcode{MAIN_PATH_INVERSION/ITERATION_STEP_PATH} which contains the 
files with synthetic data. Make sure the path ends on ``/".
%- - - - - - - - - - - - - - - - - - - - - - - - - - - - -
\subsubsection{\lcode{PATH_KERNEL_REFERENCE_MODELS}} 
This path and any contained files are only relevant for \lcode{USE_PATH_SPECIFIC_MODELS = .true.}!
Subdirectory of current iteration step path
\lcode{MAIN_PATH_INVERSION/ITERATION_STEP_PATH} in which path-specific kernel reference models are expected.
The reference model files are assumed to have names like \lcode{krm_EVENTID_STATIONNAME}.
Make sure the path ends on ``/".

%
%++++++++++++++++++++++++++++++++++++++++++++++++++++++++++
\section{Event List File} \label{files,sec:event_list}
%++++++++++++++++++++++++++++++++++++++++++++++++++++++++++
%
Please find the template event list file \lcode{template/file_event_list_template}.

\begin{itemize}
\item first line contains single character ``C'' or ``S'', defining the coordinate system (``C''artesian or ``S''pherical)
  with respect to which the given event coordinates lat,lon are interpreted
\item each following non-empty line of the file is interpreted as a definition of one event and must 
  contain the following space-separated values:
  \begin{itemize}
  \item[eventid]  13 character name, (e.g. \lcode{2006.10.2977} or \lcode{061113_141238}) 
    should \emph{not} contain whitespace!
  \item[origintime]  characters of form \lcode{yyyymmdd_hhmmss_nnnnnnnnn} or \lcode{yyyymmdd_hhmmss}
    (i.e.\ with or without nano-seconds), e.g. \lcode{20130320_170012} or 
    \lcode{20130320_170002_718000000}
  \item[lat] latitude in degrees, \lcode{-90 <= lat <= 90} (``S'') or first coordinate in 
    wavefield points / inversion grid - frame (``C'') $\rightarrow$  read the section on inversion grid definitions 
    (\ref{basic_steps,sec:invgrid})
  \item[lon] longitude in degrees, \lcode{0 <= lon <= 360} (``S'') or second  coordinate in 
    wavefield points / inversion grid -frame (``C'') (read \ref{basic_steps,sec:invgrid})
  \item[depth] source depth in km (``S''), or third coordinate in wavefield points / inversion grid -frame (``C'') 
    (read \ref{basic_steps,sec:invgrid})
  \item[mag] factor on source mechanism
  \item[typ] source type:  0 = force, 1 = moment tensor, -1: not specified
  \item[mom/frce] either 3 values (force vector) or 6 values (moment tensor)
  \end{itemize}
\end{itemize}
%
%++++++++++++++++++++++++++++++++++++++++++++++++++++++++++
\section{Station List File} \label{files,sec:station_list}
%++++++++++++++++++++++++++++++++++++++++++++++++++++++++++
%
Please find the template station list file \lcode{template/file_station_list_template}.

\begin{itemize}
\item first line contains single character ``C'' or ``S'', defining the coordinate system (``C''artesian or ``S''pherical)
  with respect to which the given event coordinates lat,lon are interpreted
\item each following non-empty line of the file is interpreted as a definition of one station and must 
  contain the following space-separated values:
  \begin{itemize}
  \item[station\_name] 5 character name, which should \emph{neither} contain whitespace \emph{nor} underscors ``\_''!
  \item[network\_code] 6 character network code
  \item[lat] latitude in degrees, \lcode{-90 <= lat <= 90} (``S'') or first coordinate in 
    wavefield points / inversion grid - frame (``C'') $\rightarrow$  read the manual on inversion grid definitions 
    (\ref{basic_steps,sec:invgrid})
  \item[lon] longitude in degrees, \lcode{0 <= lon <= 360} (``S'') or second  coordinate in 
    wavefield points / inversion grid -frame (``C'') (read \ref{basic_steps,sec:invgrid})
  \item[elevation] altitude of station (``S''), or third coordinate in wavefield points / inversion grid -frame (``C'') (read \ref{basic_steps,sec:invgrid})
  \end{itemize}
\end{itemize}
%
%++++++++++++++++++++++++++++++++++++++++++++++++++++++++++
\section{Measured Data Files} \label{files,sec:measured_data}
%++++++++++++++++++++++++++++++++++++++++++++++++++++++++++
%
All measured data files are expected to be located in the directory \lcode{PATH_MEASURED_DATA} as defined
in the main parameter file.

One measured data file contains all data values for one specific station component and a specific event.
Its filename is by convention \lcode{data_EVENTID_STATIONNAME_COMP}

The files are text files containing $1$ column of \lcode{MEASURED_DATA_NUMBER_OF_FREQ} complex numbers, 
which can be understood by \lcode{FORTRAN} \lcode{read} command, i.e.\ of form \lcode{( real_part , imag_part )} .

Line \lcode{i} contains the measured data value for the \lcode{i}\textsuperscript{th} frequency, as defined by vector
of indices \lcode{MEASURED_DATA_INDEX_OF_FREQ} and frequency step \lcode{MEASURED_DATA_FREQUENCY_STEP}. \\
In particular, this means that \emph{all} measured data files must contain the \emph{same} frequency discretization, given
by parameters \lcode{MEASURED_DATA_INDEX_OF_FREQ}, \lcode{MEASURED_DATA_FREQUENCY_STEP} of the main 
parameter file.
%
%++++++++++++++++++++++++++++++++++++++++++++++++++++++++++
\section{Event (Station) Filter Files} \label{files,sec:filters}
%++++++++++++++++++++++++++++++++++++++++++++++++++++++++++
%
All event (or station) filter files are expected to be located in the directory \lcode{PATH_EVENT_FILTER} 
(or \lcode{PATH_STATION_FILTER}), respectively. Both paths savely be set to the same directory.

A filter file contains all spectral filter values associated with a specific event (or station component). 
The file names are by convention \lcode{filter_EVENTID} (or \lcode{filter_STATIONNAME_COMP}), respectively.

In the same way as measured data files, the filter files are text files containing $1$ column of 
\lcode{MEASURED_DATA_NUMBER_OF_FREQ} complex numbers, 
which can be understood by \lcode{FORTRAN} \lcode{read} command, i.e.\ of form \lcode{( real_part , imag_part )} .

Line \lcode{i} contains the filter value for the \lcode{i}\textsuperscript{th} frequency, as defined by vector
of indices \lcode{MEASURED_DATA_INDEX_OF_FREQ} and frequency step \lcode{MEASURED_DATA_FREQUENCY_STEP}. \\
In particular, this means that \emph{all} filter files must contain the \emph{same} frequency discretization, given
by parameters \lcode{MEASURED_DATA_INDEX_OF_FREQ}, \lcode{MEASURED_DATA_FREQUENCY_STEP} of the main 
parameter file.
%
%++++++++++++++++++++++++++++++++++++++++++++++++++++++++++
\section{Synthetic Data Files} \label{files,sec:synth_data} 
%++++++++++++++++++++++++++++++++++++++++++++++++++++++++++
%
All synthetic data files are expected to be in the directory \lcode{PATH_SYNTHETIC_DATA} as defined 
in the parameter file of the current iteration step.

One synthetic data file contains the complete synthetic data values for one specific combination of
event, station and station component. Its filename is by convention \lcode{synthetics_EVENTID_STATIONNAME_COMP}

The files are text files containing $1$ column of \lcode{ITERATION_STEP_NUMBER_OF_FREQ}  
complex numbers, which can be understood by \lcode{FORTRAN} \lcode{read} command, i.e.\ of form \lcode{( real_part , imag_part )} .

Line \lcode{i} contains the synthetic data value for the \lcode{i}\textsuperscript{th} frequency, as defined by vector
of indices \lcode{ITERATION_STEP_INDEX_OF_FREQ} and frequency step \lcode{MEASURED_DATA_FREQUENCY_STEP}.
%
%++++++++++++++++++++++++++++++++++++++++++++++++++++++++++
\section{Vtk Files} \label{files,sec:vtk_files}
%++++++++++++++++++++++++++++++++++++++++++++++++++++++++++
%
For visualization of basic objects of the inversion, such as the inversion grid, 
the wavefield points, the integration weights etc., as well as some inversion results
and models, \ASKI{} uses \lcode{vtk} files in the old ``legacy'' format.

These files contain \emph{both}, geometry
information and data values on geometric objects. The format can be either \lcode{ASCII} or 
\lcode{BINARY}, controlled by flag \lcode{DEFAULT_VTK_FILE_FORMAT} in main parfile.\\
General information on the file format may be found under \url{www.vtk.org/VTK/img/file-formats.pdf}

\ASKI{} uses separate software modules for writing \lcode{vtk} based on wavefield points, or 
inversion grids or event-station path lines (or event, station coordinates). 
%
%++++++++++++++++++++++++++++++++++++++++++++++++++++++++++
\section{Data and Model Space File} \label{files,sec:dmspace}
%++++++++++++++++++++++++++++++++++++++++++++++++++++++++++
%
Files in which a data and model space is defined have the following form. \emph{It might be very instructive 
to also have a look at the provided example template files} \lcode{template/data_model_space_info_template_*} .

The blocks described in the subsections below should be put into a text file, one after another, in arbitrary order, 
i.e.\ it does not matter which block comes first (DATA SAMPLE or MODEL VALUES). It is
also supported to provide only one of the blocks, if the other one is not required for some application. 

\emph{There can be arbitrary commentary in front of a block, between blocks and after the blocks.} However, 
any line starting with \lcode{DATA SAMPLES} or \lcode{MODEL VALUES} is interpreted as the starting line
of the respective block. Inside the blocks, ther must not be unexpected lines!

%The header block must come first, then the data space block and model parameter block. The order of the latter two is arbitrary, both orders are allowed. %% IN CASE OF INTRODUCING SOME HEADER BLOCK IN THE FUTURE

%% %---------------------------------------------------------
%% \subsection{Header Block}
%% %---------------------------------------------------------
%% {\bf line 1}: currently ignored (file format version specification possible, header comment)

%% {\bf line 2}: must either contain \lcode{ASCII} or \lcode{BINARY}\\
%% At the moment, this file must be a formatted text file (nothing else supported yet). In the future, also 
%% binary or mixed text/binary formats could be supported (similar to e.g. vtk)

%---------------------------------------------------------
\subsection{Model Values Block}
%---------------------------------------------------------

{\bf line}: \lcode{MODEL VALUES}\\
this line defines that the model values block (definition of the model values) starts here. 

{\bf line}: \lcode{INVERSION_GRID_CELLS value}\\
where \lcode{value} is either \lcode{ALL} (all inversion grid cells are taken) or \lcode{SPECIFIC} 
(specific definition of set of invgrid cells following below)

{\bf line}: \lcode{PARAMETERS value}\\
where \lcode{value} is either \lcode{ALL} (all inversion grid cells are taken) or \lcode{SPECIFIC} 
(specific definition of model parameters for each invgrid cell, following below. Only allowed if 
\lcode{INVERSION_GRID_CELLS SPECIFIC}). 

{\bf If \lcode{PARAMETERS ALL}, line}: \lcode{nparam param_1 ... param_n}\\
defines the parameters used for all inversion grid cells (assuming \lcode{MODEL_PARAMETRIZATION} defined in main parameter file).

If \lcode{INVERSION_GRID_CELLS SPECIFIC}, the following line must contain the number of cells \lcode{ncell} 
which should be taken, followed by \lcode{ncell} blocks of lines, each defining an inversion grid cell. In case
of \lcode{PARAMETERS ALL}, these blocks consist of a single line line containing an inversion grid cell index.
In case of \lcode{PARAMETERS SPECIFIC}, these blocks consist of two lines: one line containing an inversion grid cell
index and an additional second line of the form\\
\lcode{nparam param_1 ... param_n} \\
defining the parameters to be used for this specific inversion grid cell (assuming \lcode{MODEL_PARAMETRIZATION} defined in main parameter file).

%---------------------------------------------------------
\subsection{Data Samples Block}
%---------------------------------------------------------
line \lcode{DATA SAMPLES}\\
this line defines that the data samples block starts (definition of data samples) here. 

line of form: \lcode{WEIGHTING value}, where value is one of is one of \lcode{NONE}, \lcode{BY_PATH}, 
\lcode{BY_FREQUENCY}, \lcode{BY_PATH_AND_FREQUENCY} (by defining such weights, you might account for event clustering or huge differences in magnitudes of the events) \\
In case of \lcode{NONE}, all data weights are internally set to 1.0 (i.e. no actual weighting is performed).
Values \lcode{BY_PATH} and \lcode{BY_PATH_AND_FREQUENCY} are only allowed in case of \lcode{PATH SPECIFIC}, in which case
the pairs \lcode{evid staname} in a specific path definition (below) is expected to be followed by one number $>$ 0.0 and $\le$ 1.0 (the weight).
The frequency dependent weighting values (in cases \lcode{BY_FREQUENCY}, \lcode{BY_PATH_AND_FREQUENCY}) are defined
in a separate line following the lines of form \lcode{nfreq ifreq_1 ... ifreq_n} (for either case of \lcode{FREQUENCIES ALL}
or \lcode{FREQUENCIES SPECIFIC}). This separate lines have themselves the form \lcode{nfreq w_1 ... w_n} defining \lcode{nfreq} weights
in range $>$ 0.0 and $\le$ 1.0. 
In case \lcode{BY_PATH_AND_FREQUENCY}, both weights (for path and frequency) are \emph{multiplied} for each data sample.

line of form: \lcode{PATHS value}, where value is either \lcode{ALL} (all paths for a given set of event and station indices
are used), or \lcode{SPECIFIC} (a specific definition of paths as a series of event and station index pairs follows below)

If \lcode{PATHS ALL}, the next two lines are of form \lcode{nev iev_1 ... iev_n} and \lcode{nstat istat_1 ... istat_n}, defining the set
of event and station indices, which form (by all combinations) the used paths. 

line of form: \lcode{COMPONENTS value}, where value is either \lcode{ALL} (for all paths, the same components are used) or \lcode{SPECIFIC}
(only allowed if \lcode{PATHS SPECIFIC}, for each path a specific set of components may be defined)

If \lcode{COMPONENTS ALL}, the next line is of form \lcode{ncomp comp_1 ... comp_n} defining the component indices for all paths.

line of form: \lcode{FREQUENCIES value}, where value is either \lcode{ALL} (for all paths, the same frequency indices are used) or \lcode{SPECIFIC}
(only allowed if \lcode{PATHS SPECIFIC}, for each path a specific set of frequency indices may be defined)

If \lcode{FREQUENCIES ALL}, the next line is of form \lcode{nfreq ifreq_1 ... ifreq_n} defining the frequency indices for all paths.

line of form: \lcode{IMRE value}, where value is either \lcode{ALL} (for all paths, the same set of imaginary/real parts are used) or \lcode{SPECIFIC}
(only allowed if \lcode{PATHS SPECIFIC}, for each path a specific set of imaginary/real parts may be defined)

If \lcode{IMRE ALL}, the next line is of form \lcode{nimre imre_1 ... imre_n} defining imaginary (i.e. \lcode{imre_i} = \lcode{im}) or real parts 
(\lcode{imre_i} = \lcode{re}) for all paths.

If \lcode{PATHS SPECIFIC}, the following line must contain the number \lcode{npaths} of paths which should be used, followed by 
npahts blocks of lines, each defining the path and the data samples for that path. \\
These blocks constist of at least one line containing the event /station index pair \lcode{iev istat}. \\
For each keyword \lcode{COMPONENTS}, \lcode{FREQUENCIES} and \lcode{IMRE} -- if \lcode{SPECIFIC} -- one line is added to such a block 
of lines, in the same form as the line following \lcode{keyword ALL} (see above), 
defining the specific components, frequencies or set of imaginary/real parts for each of the specific paths.
%
%++++++++++++++++++++++++++++++++++++++++++++++++++++++++++
\section{\lcodetitle{ecartInversionGrid} Files} \label{files,sec:ecart_invgrid}
%++++++++++++++++++++++++++++++++++++++++++++++++++++++++++
%
%---------------------------------------------------------
\subsection{Nodes Coordinates Files}
%---------------------------------------------------------
These files contain a collection of points in space, given in Cartesian \lcode{X}-, \lcode{Y}-, 
\lcode{Z}-coordinates. They must be text files and have the following format.

The first line contains a single integer value, indicating the number of lines to come (i.e.\ the 
number of points).\\
Each following line contains 3 floating point numbers (separated by white space) defining Cartesian 
\lcode{X}-, \lcode{Y}-, \lcode{Z}-coordinates of a point.

%---------------------------------------------------------
\subsection{Cell Connectivity Files}
%---------------------------------------------------------
These files contain the definition of cells, based on points as defined in the nodes coordinates files.
They must be text files and have the following format.

The first line contains a single integer value, indicating the number of lines to come (i.e.\ the 
number of cells).\\
Each following line contains \lcode{n} integer numbers (separated by white space, \lcode{n = 4} 
in case of tet4-type cells, \lcode{n = 8} in case of hex8-type cells), which define the control nodes 
of the cell and correspond to the point indices in the respective nodes coordinates file, whereby the 
lowest point index is 1, corresponding to the second line (first point) in the nodes coordinates file.

The order of the point indices in a line is assumed to correspond to the vtk cell conventions!
In case one of the cell connectivity files not existing, or their first line containing value 0, no cells
of the respective type will be created.

%---------------------------------------------------------
\subsection{Cell Neighbours File}
%---------------------------------------------------------
The terminology ``lines'' below refers to the case of this file not being binary, but a text file. 
In case of this file being binary, the file content is expected value by value as on the rows of the
text file. It will be opened by \lcode{FORTRAN} code with attribute \lcode{access='stream'}
(i.e.\ expecting the values as a simple byte stream) and expects integer values of \lcode.

The first line contains the total number of inversion grid cells \lcode{ncell}.\\
The next \lcode{ncell} lines (one for each cell in order of the cell index) are of the form:\\
\lcode{nnb icell_1 ... icell_nnb}\\
whereby \lcode{nnb} is the number of neighbours of the respective cell (must be \lcode{0} if 
no neighbours) followed by \lcode{nnb} cell indices \lcode{icell_1 ... icell_nnb}, defining 
the neighbour cells, if there are any neighbours.
%
%++++++++++++++++++++++++++++++++++++++++++++++++++++++++++
\section{GSHHS native binary shore line files} \label{files,sec:GSHHS_bin}
%++++++++++++++++++++++++++++++++++++++++++++++++++++++++++
%
The files that are meant here, are GSHHG files containing shore line data in native binary format.
The files can be downloaded e.g.\ via 
\url{https://www.ngdc.noaa.gov/mgg/shorelines/data/gshhg/latest/}, choosing the package probably named 
like \lcode{gshhg-bin-?.?.?.zip}, extracting the shore line files
\lcode{gshhs_c.b}, \lcode{gshhs_l.b}, \lcode{gshhs_i.b}, \lcode{gshhs_h.b}, \lcode{gshhs_f.b}.

\ASKI{} expects the files to contain 4-byte integers in big-endian. The file is written sequentially, 
information of one the shore line polygons one after another. 

For each polygon, there is a header consisting of 11 4-byte signed integers. The first entry of the header
is the polygon index (in consecutive numbering, starting from 0). The second entry of the header gives
the number of points contained in the polygon (the point information follows after the header). 
All other header information are ignored by \ASKI{}.

After reading a polygon header, \ASKI{} reads the expected number of points for that polygon (as given by second
entry of header). Each point is a pair of two 4-byte signed integers representing longitude and latitude 
(in that order) in the unit of micro-degrees.

This native binary format GSSHS shore line files is documented, e.g.\ at the end of file 
\url{https://www.ngdc.noaa.gov/mgg/shorelines/data/gshhg/latest/readme.txt}):
{\scriptsize
\begin{verbatim}
   struct GSHHG {  /* Global Self-consistent Hierarchical High-resolution Shorelines */
         int id;         /* Unique polygon id number, starting at 0 */
         int n;          /* Number of points in this polygon */
         int flag;       /* = level + version << 8 + greenwich << 16 + source << 24 + river << 25 */
         /* flag contains 5 items, as follows:
          * low byte:    level = flag & 255: Values: 1 land, 2 lake, 3 island_in_lake, 
                                                     4 pond_in_island_in_lake
          * 2nd byte:    version = (flag >> 8) & 255: Values: Should be 12 for GSHHG release 12 
                                                              (i.e., version 2.2)
          * 3rd byte:    greenwich = (flag >> 16) & 1: Values: Greenwich is 1 if Greenwich is crossed
          * 4th byte:    source = (flag >> 24) & 1: Values: 0 = CIA WDBII, 1 = WVS
          * 4th byte:    river = (flag >> 25) & 1: Values: 0 = not set, 1 = river-lake and level = 2
          */
         int west, east, south, north;   /* min/max extent in micro-degrees */
         int area;       /* Area of polygon in 1/10 km^2 */
         int area_full;  /* Area of original full-resolution polygon in 1/10 km^2 */
         int container;  /* Id of container polygon that encloses this polygon (-1 if none) */
         int ancestor;   /* Id of ancestor polygon in the full resolution set that was the source of 
                                                                        this polygon (-1 if none) */
 };

 Following each header structure is n structures of coordinates:

 struct GSHHG_POINT {	/* Each lon, lat pair is stored in micro-degrees in 4-byte signed integer format */
 	int32_t x;
 	int32_t y;
 };

 Some useful information:

 A) To avoid headaches the binary files were written to be big-endian.
    If you use the GMT supplement gshhg it will check for endian-ness and if needed will
    byte swab the data automatically. If not then you will need to deal with this yourself.

 B) In addition to GSHHS we also distribute the files with political boundaries and
    river lines.  These derive from the WDBII data set.

 C) As to the best of our knowledge, the GSHHG data are geodetic longitude, latitude
    locations on the WGS-84 ellipsoid.  This is certainly true of the WVS data (the coastlines).
    Lakes, riverlakes (and river lines and political borders) came from the WDBII data set
    which may have been on WGS072.  The difference in ellipsoid is way less then the data
    uncertainties.  Offsets have been noted between GSHHG and modern GPS positions.

 D) Originally, the gshhs_dp tool was used on the full resolution data to produce the lower
    resolution versions.  However, the Douglas-Peucker algorithm often produce polygons with
    self-intersections as well as create segments that intersect other polygons.  These problems
    have been corrected in the GSHHG lower resolutions over the years.  If you use gshhs_dp to
    generate your own lower-resolution data set you should expect these problems.

 E) The shapefiles release was made by formatting the GSHHG data using the extended GMT/GIS
    metadata understood by OGR, then using ogr2ogr to build the shapefiles.  Each resolution
    is stored in its own subdirectory (e.g., f, h, i, l, c) and each level (1-4) appears in
    its own shapefile.  Thus, GSHHS_h_L3.shp contains islands in lakes for the high res
    data. Because of GIS limitations some polygons that straddle the Dateline (including
    Antarctica) have been split into two parts (east and west).

 F) The netcdf-formatted coastlines distributed with GMT derives directly from GSHHG; however
    the polygons have been broken into segments within tiles.  These files are not meant
    to be used by users other than via GMT tools (pscoast, grdlandmask, etc).
\end{verbatim}
}
