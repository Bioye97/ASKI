% -*-LaTex-*-

%-----------------------------------------------------------------------------
%   Copyright 2016 Florian Schumacher (Ruhr-Universitaet Bochum, Germany)
%
%   This file is part of the ASKI manual as a LaTeX document with main file
%   manual.tex
%
%   Permission is granted to copy, distribute and/or modify this document
%   under the terms of the GNU Free Documentation License, Version 1.3
%   or any later version published by the Free Software Foundation;
%   with no Invariant Sections, no Front-Cover Texts, and no Back-Cover Texts.
%   A copy of the license is included in the section entitled ``GNU
%   Free Documentation License''. 
%-----------------------------------------------------------------------------
%
\ASKI{} is a modularized software package written by the main authors Florian Schumacher and
Wolfgang Friederich (Ruhr-University Bochum, Germany). It offers sensitivity and regularization 
analysis tools for seismic datasets
as well as a scattering-integral-type full waveform inversion (FWI) concept based on waveform sensitivity 
kernels (i.e.\ Fr\'echet derivatives of seismic waveform data functionals) derived from Born scattering theory,
having Gauss-Newton convergence properties.
This inversion concept is presented in our paper \cite{Schumacher16} and in Florian's doctoral
dissertation \cite{_743d334d-dfa4-4a16-8cc5-91cdadc95271}.

\ASKI{} does not implement an intrinsic code for simulation of seismic wave propagation in order to 
solve the forward problem, but instead provides generalized interfaces to different external wave 
propagation codes. At the moment, the 1D semi-analytical code Gemini \cite{friederich_wd1995} and the 3D spectral 
element code SPECFEM3D \cite{TrKoLi08} are supported in \emph{both}, Cartesian and spherical framework. 
Additionally \ASKI{} supports the 3D nodal discontinuous Galerkin code NEXD \cite{Lambrecht.2015}
and plan to implement support for the Finite Difference code SOFI \cite{bohlen2002parallel}
in the near future. 
Please do not hesitate to add your own forward code, we are happy to support you during that process.

The resulting very modular concept of \ASKI{} allows us to keep the spatial description of the inverted 
models completely independent of the spatial description of the model used for solving the forward 
problem. This way, the overall iverse problem can be approached and regularized more physically compared
with using the forward grid also for inversion. The normally very fine spatial discretization of the model
domain is employed by the forward solver. This discretiation is usually weigh too fine to be used
directly for inversion since usually seismic data can only resolve much coarser structures and the
resolution also varies throughout the model domain. 
The separation concept of \ASKI{} \emph{naturally} alows to choose the spatial model
resolution heterogeneously and \emph{anew} in each iteration of FWI.

Instead of using time-dependent values of ground motion (i.e.\ samples of a time-series of seismic data), 
\ASKI{} uses freqency-dependent complex values of ground motion, recorded at a certain receivers excited 
by a certain seismic sources. This, mainly, has reasons of computational feasibility.
For time-domain forward methods, we implement an on-the-fly Fourier Transform in order to provide the
required spectral waveforms. 

Using sensitivity kernels $K$, data residuals $\delta d_i$ are connected to model uptdates $\delta m$ by 
an integral relation $\delta d_i = \int_{\text{earth}} \delta \vec{m} \cdot \vec{K}_i$. In order to build a linear 
system, the model update $\delta \vec{m}$ is assumed to be constant throughout small scattering volumes 
$\Omega_j$, where $\Omega = \overset{\centerdot}{\bigcup}_j \Omega_j$. These volumes constitute the cells of the volumetric 
inversion grid and the sensitivity matrix contains entries of preintegrated kernels $\int_{\Omega_j} K_i$.

The sensitivity kernels $K$ are computed from forward wavefields and strains produced on a set of points in 
space, which
is dependent on the particular forward method. This set of points is refered to as \emph{wavefield points}. The 
wavefields are written to file, by the respective forward method, which may require large discspace. 
Providing methods for constructing quadrature rules for arbitrary point sets contained in 
arbitrary volumes, 
\ASKI{} computes integration weights for integration of functions on the wavefield points over the volumetric
cells of the inversion grid. The inversion grid takes care of the localization of wavefield points 
inside the inversion grid cells and, if requested, the transformation of cells to a hexahedral (or tetrahedral) 
standard 
cell for the computation of the integration weights. Hence, some combinations of wavefield points (i.e.\ 
forward methods), integration weight types and inversion grid types are not possible. 

The preintegrated kernel values are also written to files, which may be flexibly read in for arbitrary 
subsets of data
by the independent binary programs conducting any sensitivity analysis 
or an interation step in the iterative full waveform inversion.
Those tools work on the sensitivity matrix, which in the FWI is used in a linear system of equations which
relates data residua $\delta d_i$ to a model improvement $\delta m$. 
In the course of the iterative full waveform inversion, it is naturally possible by \ASKI{} to gradually increase
the frequency content of the inverted data and to choose smaller inversion grid cells in each iteration, i.e.\ increasing
the spatial resolution dependent on frequency.


%% \vspace*{1cm}
%% {\bf {\Huge TODO} PUT A GENERAL OVERWIEW OF ASKI HERE, SO THAT THE SECTIONS IN ``BASIC STEPS'' DO NOT NEED TO 
%% EXPLAIN EVERY BIT OF BASIC STUFF: in the following there is a collection of unsorted bits of documentation
%% which could be involved in this section\\
%% also include some pictures: a set of points inside the volumes of the inversion grid, a sensitivity kernel
%% for one specific datum (real part of some frequency sample of a seismogram (src,rec,comp)) explaining that 
%% positive high values at one scatterer mean that this very datum increases (significantly) if the current 
%% model parameter is increased at this scatterer}

%% All operations \ASKI{} can conduct are based on Waveform Sensitivity Kernels. 

%% IN ESSENCE, PUT SOME SHORTENED, MORE INTUITIVELY UNDERSTANDABLE VERSION OF THE \ASKI{} PAPER (when existend) 
%% HERE, WHICH EXPLAINS THE TERMINI ``SENSITIVITY KERNELS'', ``INERSION GRID'', ``WAVEFIELD POINTS'', ``INTEGRATION WEIGHTS''
%% AND GIVES THE PRINCIPLE IDEAS OF COMPUTING THE KERNELS FROM WAVEFIELD GIVEN ON THE WAVEFIELD POINTS, INTEGRATING THEM ONTO THE
%% INVERSION GRID (hence, connecting forward simulation = data with the model) AND SOLVING LINEAR SYSTEMS 
%% CONTAINING THE SENSITIVITY VALUES. IT SHOULD ALSO BE EXPLAINED THAT ASKI USES DATA (hence, kernel sensitivity values) IN THE FREQUENCY DOMAIN AND WHY:

